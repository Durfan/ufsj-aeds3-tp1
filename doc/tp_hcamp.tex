\documentclass[12pt,a4paper]{article}

\usepackage[left=3.00cm, right=2.00cm, top=2.00cm, bottom=2.00cm]{geometry}
\usepackage{lmodern}
\usepackage[utf8]{inputenc}
\usepackage[brazil]{babel}

\usepackage{graphicx}
\usepackage{indentfirst}
\usepackage{booktabs}
\usepackage{forest}

\usepackage[numbers]{natbib}
\usepackage{url}
\bibliographystyle{plainnat}

\usepackage{setspace}
\onehalfspacing

\author{Pablo Cecilio Oliveira\\
	Alexander Cristian}
\title{Algorítimos e Estrutura de Dados III\\
	Primeiro Trabalho Prático - Hipercampos}
\date{}

\begin{document}
\maketitle

\section{Introdução}

O texto abaixo foi retirado na integra da descrição do Trabalho Prático. É necessário reescreve-lo para criar uma introdução própria ao problema proposto.

São dadas duas âncoras, dois pontos $A = (X_A , 0)$ e $B = (X_B , 0)$, formando um segmento horizontal, tal que $0 < XA < XB$ , e um conjunto $P$ de $N$ pontos da forma $(X, Y)$, tal que $X > 0$ e $Y > 0$. A figura mais à esquerda exemplifica uma possível entrada.

Para ''ligar'' um ponto $v \in P$ precisamos desenhar os dois segmentos de reta $(v, A)$ e $(v, B)$. Queremos ligar vários pontos, mas de modo que os segmentos se interceptem apenas nas âncoras. Por exemplo, a figura do meio mostra dois pontos, 1 e 4, que não podem estar ligados ao mesmo tempo, pois haveria interseção dos segmentos fora das âncoras. A figura mais à direita mostra que é possível ligar pelo menos 3 pontos, 8, 5 e 3, com interseção apenas nas âncoras. Seu programa deve computar o número máximo de pontos que é possível ligar com interseção de segmentos apenas nas âncoras.

A partir desse paragrafo será colocada uma visão geral sobre o funcionamento do programa segundo as especificações do TP mais adicionais.

\section{Implementação}

\begin{enumerate}
\item Listagem das rotinas.
\item Descrição breve dos algoritmos e das estruturas de dados utilizadas.
\item Análise da complexidade das rotinas.
\item Análise dos resultados obtidos.
\end{enumerate}

\section{Considerações finais}

Comentários gerais sobre o trabalho e as principais dificuldades encontradas em sua implementação.

\begin{flushleft}
	\nocite{*}
	\bibliography{tp_hcamp}
\end{flushleft}

\end{document}