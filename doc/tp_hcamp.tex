\documentclass[12pt,a4paper]{article}

\usepackage[left=3.00cm, right=2.00cm, top=2.00cm, bottom=2.00cm]{geometry}
\usepackage{lmodern}
\usepackage[utf8]{inputenc}
\usepackage[brazil]{babel}

\usepackage{graphicx}
\usepackage{indentfirst}
\usepackage{booktabs}
\usepackage{forest}

\usepackage[numbers]{natbib}
\usepackage{url}
\bibliographystyle{plainnat}

\usepackage{setspace}
\onehalfspacing

\author{Pablo Cecilio Oliveira\\
	Alexander Cristian}
\title{Algorítimos e Estrutura de Dados III\\
	Primeiro Trabalho Prático - Hipercampos}
\date{}

\begin{document}
\maketitle

\section{Introdução}

A computação é um instrumento muito útil para encaminhar a resolução de um problema. Podendo ele ser uma coisa simples, ou então, até mesmo algo mais elaborado. Tendo isso em vista, o primeiro trabalho prático da disciplina nos desafia a resolver o problema dos Hipercampos. O qual é visto em diversas  maratonas de programação, e pode ser resolvido de varias formas. Neste trabalho será aborada a solução julgada de melhor implementação por parte dos discentes.

\subsection{Especificação do Problema}

Em uma visão mais detalhada do problema, temos um plano cartesiano onde são dadas duas âncoras, dois pontos onde Y= 0 e os valores de $X$ variam de $A$ até $B$ formando assim um segmento de reta horizontal, tal que $0 < XA < XB$. E também se recebe como entrada um conjunto $P$ de $N$ pontos na forma $(X, Y)$ sendo $X$ e $Y$ maiores do que 0.

Ao ligar um dos pontos contidos em P às âncoras, usando segmentos de reta, formamos um triangulo, deve-se ligar vários pontos, mas de modo que eles se interceptem apenas nas ancoras. Se expressando de uma maneira mais simples, deve-se achar o maior número de triângulos contidos um dentro do outro, que se cruzam apenas na base. 

   Portanto o algoritmo trabalhado computa o número máximo de pontos que é possível ligar com interseção de segmentos apenas nas ancoras, de acordo com as entradas do usuário.

\subsection{Entrada}

A primeira linha da entrada contém três inteiros, $N(1 ≤ N ≤ 100)$, $XA$ e $XB$ $(0 < XA < XB ≤ 10000)$ representando, respectivamente, o número de pontos no conjunto $P$ e as abscissas das âncoras $A$ e $B$. As $N$ linhas seguintes contêm, cada uma, dois inteiros $Xi$ e $Yi4 $(0 < Xi, Yi ≤ 10000)$,representando as coordenadas dos pontos, para $1 ≤ i ≤ N$ . Não há pontos coincidentes e não há
dois pontos u e v distintos tais que $A$, $u$, $v$ ou $B$, $u$, $v$ sejam colineares.

\subsection{Saída}

O programa imprime uma linha contendo um inteiro, representando o número máximo de pontos de PP que podem ser ligados com interseção de segmentos apenas nas âncoras.

\subsection{Solução proposta}

O metodo em questão usado para resolver esse problema se baseia na exploração do sistema de coordenadas baricêntricas e na orientação dos segmentos de retas formados pela conexão dos pontos. Portanto em primeiro lugar é verificada a orientação das retas para isolar os casos em que elas se interceptão ou são colineares.

Logo em seguida precisa-se saber qunado um ponto esta contido em um triangulo. Uma solução simples seria traçar uma reta que segue horizontalmente para a direita, e depois fazendo comparações para saber quantas vezes ela intercepta o poligono formado, se o resultado for um número par o ponto está fora, se for impar ele está dentro. Porem isso levaria o programa a executar muitas operações, então evoluindo desse conceito chegamos a uma solução usando as coordenadas baricentricas, onde é verificado em qual lado do meio plano criado pelas arestas está o ponto.


\section{Implementação}

-Mergesort
Foi usado o mergesort para ordenar a lista encadeada pelo maior Y dos pontos. Esse algoritmo de ordenação segue o estilo dividir e conquistar, possuindo complexidade de O(n log n).

-plotGraph
Cria um aquivo "data.temp" onde são armazenados os valores de entrada, e logo em seguida escreve os comandos do "gnuplot" e manda executá-lo.

-dump 
Essa função usa um vetor de nomes de objetos R e produz representações de texto dos objetos em um arquivo ou conexão.

-soluciona
Chama as funções para formar o grafico e achar o ponto de maior valor do conjunto_t

-PQR
Verifica por meio da expreção de orientação da reta 
$(y2−y1) (x3−x2) − (y3−y2) (x2−x1)$
se os pontos são colineares, e a orientação do triangulo (horario ou anti-horario)

-solucao
Imprime a saida do problema

\section{Análise de Complexidade}

$O(N^2)$

\section{Considerações finais}

O Trabalho computacional 1 da disciplina foi uma grande oportunidade para aprender sobre grafos e LCS, que rodeiam o algoritmo otimo para a solução desse problema, o que é a introdição para programação dinamica e acreditamos ser o intuito desse trabalho, também proporcionou um contato maior com a analise de complexidade do algoritmo. 

Um dos maiores problemas no desenvolvimento foi encontrar um algoritmo que possuísse um comportamento adequado quando a entrada de valores é muito grande. Apesar da forte base matemática de nossos métodos, em alguns casos eles podem levar a uma falta de precisão, porque o sistema de números de ponto flutuante tem tamanho limitado e na maioria das vezes lida com aproximações. O problema ocorre às vezes quando um ponto p deve estar exatamente na borda de um triângulo, as aproximações levam a falhar no teste. 

Para a construção de gráficos que auxiliam em uma melhor visualização do trabalho foi necessario o Gnuplot.


\begin{flushleft}
	\nocite{*}
	\bibliography{tp_hcamp}
\end{flushleft}

\end{document}
